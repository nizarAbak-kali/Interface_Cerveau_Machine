\documentclass{beamer}

\mode<presentation>{
  %\usetheme{Warsaw}
  %\usetheme{Montpellier}
  \usetheme{PaloAlto}
  \setbeamercovered{transparent}
}

\usepackage[french]{babel}
\usepackage[latin1]{inputenc}
\usepackage[T1]{fontenc}
\usepackage{subfigure}
\usepackage{times}
\usepackage{algorithm}
\usepackage{algorithmic}

%%%%%%%%%%%%%%%%%%%%%%%%%%%%%%%%%%%%%%%%%%%%%%%%%%%%%%%%%%%%
\title{Projet Tuteur� : BCI}
\subtitle{Interface Cerveau-Machine pour d�tection des diff�rents �tats d'une personne atteinte de la maladie d'Alzheimer}

\author{Nizar ABAK-KALI\\Alexis ZURAWSKA}

\institute{
  PARIS XIII Saint-Denis \\
  nizarabakkali93@gmail.com 
  alexis.zurawska@yahoo.fr
  }

\date{\today}
 

\pgfdeclareimage[height=0.8cm]{le-logo}{liasd}
\logo{\pgfuseimage{le-logo}}

\AtBeginSection[]{
  \begin{frame}<beamer>{PLAN DE LA PRESENTATION}
    \tableofcontents[currentsection]
  \end{frame}
}

%%%%%%%%%%%%%%%%%%%%%%%%%%%%%%%%%%%%%%%%%%%%%%%%%%%%%%%%%%%%
\begin{document}

\begin{frame}
  \titlepage
\end{frame}

\begin{frame}{PLAN DE LA PRESENTATION}
  \tableofcontents
\end{frame}

%%%%%%%%%%%%%%%%%%%%%%%%%%%%%%%%%%%%%%%%%%%%%%%%%%%%%%%%%%%%
\section{Introduction}

%%%%%%%%%%%%%ETAT DES LIEUX
\begin{frame}  
  \begin{exampleblock}{Pr�sentation du Projet}
        Pouvoir conna�tre, � partir de l'EEG, l��tat d'un patient atteint d'Alzheimer.\\
        5 �tats :
        \begin{itemize}
            \item lecture 
            \item �criture
            \item attention sur un film
            \item discussion
            \item repos 
        \end{itemize}
    \end{exampleblock}
\end{frame}


%%%%%%%%%%%%%%%Problematique rencontre
\begin{frame}
  \begin{alertblock}{Probl�matique rencontre}
    \begin{itemize}
    \item Comment classer les signaux r�cup�rer ?
    \item Quels crit�res/caract�ristiques doit-on prendre en compte pour le classification ?
    \item Comment rendre notre syst�me plus performant?
    \end{itemize}
  \end{alertblock}
\end{frame}

%%%%%%%%%%%%%%%%%%%%%%%%%%%%%%%%%%%%%%%%%%%%%%%%%%%%%%%%%%%%
\section{Outils Utilises}
\begin{frame}
  \begin{exampleblock}{Outils}
   \begin{itemize}
    \item Langage : Java 
    \item IDE : IntelliJ
    \item Librairie JFreeCharts : G�n�ration de Graphiques
    \item Github 
    \item LaTex 
    \end{itemize}
\end{exampleblock}
\end{frame}
%%%%%%%%%%%%%%%%%%%%%%%%%%%%%%%%%%%%%%%%%%%%%%%%%%%%%%%%%%%
\section{Avancement}
\begin{frame}
  \begin{exampleblock}{Ce qui a �t� fait}
   \begin{itemize}
    \item  Mise en place du d�p�t GitHub avec toutes les sources et certains documents et liens utiles
    \item D�veloppement des fonctions math�matiques
    \item Squelette du rapport final
    \end{itemize}
\end{exampleblock}
\end{frame}

\begin{frame}
  \begin{exampleblock}{Ce qu'il reste a faire}
   \begin{itemize}
    \item D�veloppement du traitement du signal EEG
    \item D�veloppement des fonction temporelles pour le réseau de neurones
    \item D�veloppement du Classifier
    \end{itemize}
\end{exampleblock}
\end{frame}

%%%%%%%%%%%%%%%%%%%%%%%%%%%%%%%%%%%%%%%%%%%%%%%%%%%%%%%%%%%%
\begin{frame}
  \titlepage
\end{frame}
%%%%%%%%%%%%%%%%%%%%%%%%%%%%%%%%%%%%%%%%%%%%%%%%%%%%%%%%%%%%
\end{document}


