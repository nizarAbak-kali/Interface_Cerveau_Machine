\part{Introduction Théorique} % (fold)
\label{prt:introduction_ _théorique_}
	
	\chapter{BCI} % (fold)
	\label{chap:bci}
	
	BCI: Brain Computer Interface (ou Interface Cerveau-Machine en français).
	
	Aussi appelée Interface Neuronale Directe (abrégée IND), il s'agit d'une interface de communication directe entre un cerveau et un composant externe (généralement un ordinateur).
	Ce procédé consiste en fait à restaurer, au moins partiellement des facultés perdues, voire non connues (dans le cas d'un handicap de naissance). C'est le cas notamment d'une personne ayant perdu totalement la vue à qui on a implanté un tel procédé au niveau de son cortex visuel, ce qui lui a permis de percevoir de nouveau la lumière. Si ces procédés ne sont pas miraculeux pour le moment, ils peuvent également permettre d'effectuer des actions par l'intermédiaire de la pensée. Ainsi, des personnes ont pu écrire sur un écran d'ordinateur ou déplacer le curseur d'une souris en imaginant simplement l'action de le faire. D'autres ont pu déplacer un bras robotisé pour qu'il leur ramène quelque chose par exemple.
	Ce procédé peut être unidirectionnel (la machine envoie des données au cerveau ou le contraire mais pas les deux en même temps) ou bidirectionnel.
	Ces interfaces neuronales directes présentent toutefois plusieurs limites:
	\begin{itemize}
		\item [-] chacune d'elles ne fait qu'une tâche précise et non plusieurs;
		\item [-] elles étaient initialement développées dans un but médical rendant l'accès difficile au grand public;
		\item [-] à cela s'ajoute le fait que les populations défavorisées n'auront pas les moyens de se procurer de tels équipements, le prix étant très élevé.
	\end{itemize}

	
	
	
	% chapter bci (end)

	\chapter{Généralité} % (fold)
	\label{chap:généralité}
	
	Ici, deux concepts clés doivent être définis pour comprendre en quoi consiste notre projet: les réseaux de neurones ainsi que l'Electroencéphalographie (EEG). 
	
	D'abord, il est bon de voir un réseau de neurones artificiels comme un modèle de calcul représentant schématiquement les réseaux de neurones biologiques. C'est à celui-ci que l'on va imposer l'apprentissage d'un échantillon de données. Pour cela, nous alllons utiliser l'algorithme du gradient, le gradient étant, en mathématiques, un vecteur qui représente la variation d'une fonction en fonction de la variation de ses paramètres. Tout comme nous, le réseau de neurones n'est pas parfait et fera des erreurs lors de son apprentissage. Il faudra donc utiliser la rétropropagation du gradient qui est un algorithme servant à corriger les erreurs du réseau de neurones pour qu'il ne lest reproduise pas.
	
	Il est également important de définir l'EEG: méthode d'exploration cérébrale mesurant l'activité du cerveau au moyen d'électrodes posés sur le cuir chevelu. On l'appelle aussi électroencéphalogramme et il est comparable à l'électrocardiogramme.
	
	% parler des IHM vue en reunion 

	% chapter généralité (end)

	\chapter{Contexte} % (fold)
	\label{chap:contexte}
	
	   Dans le cadre de ce projet, nous devions développer un réseau de neurones afin de reconnaître les différents états d'une personne atteinte de la maladie d'Alzheimer. Cinq états devaient être détectés:
	   \begin{itemize}
	   	\item[-] lecture;
		\item[-] attention sur un film;
		\item[-] écriture;
		\item[-] discussion;
		\item[-] repos.
	   \end{itemize}
	
	
	
	
		% Contexte et Problématique 
	% chapter contexte (end)

% part introduction_ _théorique_ (end)