\part*{Introduction}

 Durant notre première année de master en informatique à l'université Paris 8 Vincennes - Saint-Denis, nous devions développer des applications au cours d'un module intitulé "Projets tuteurés".
 Ce travail devait être effectué sur toute l'année et faire l'objet d'une présentation en plus de ce rapport-ci. Nous devions choisir une thématique parmi celles proposées.
 Ensuite, notre tuteur attitré nous attribuait un projet en nous expliquant ce qu'il attendait de nous. En ce qui nous concerne, nous devions développer un réseau de neurones capable de reconnaître différents d'activité d'une personne de par son activité cérébrale.
 C'est après vous avoir présenté certaines généralités touchant au BCI, puis sur l'utilité et l'implémentation des réseaux de neurones et des interfaces cerveaux-machines que nous exposerons l'état de l'art en ce qui concerne la détection et la classification d'activités cérébrales. Puis, nous expliquerons le fonctionnement du système. Ensuite, nous détaillerons l'architecture de l'application. Enfin nous conclurons sur les objectifs atteints et les prévisions futures .  