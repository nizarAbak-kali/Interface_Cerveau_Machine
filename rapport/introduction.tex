\part*{Introduction}

 Durant notre première année de master en informatique à l'université Paris 8 Vincennes - Saint-Denis, nous devions développer des applications au cours d'un module intitulé "Projets tuteurés".
 Ce travail serait effectué sur toute l'année et doit faire l'objet d'une présentation en plus de ce rapport-ci. Nous devions choisir une thématique parmi celles proposées.
 Ensuite, notre tuteur attitré nous attribuait un projet en nous expliquant ce qu'il attendait de nous. En ce qui nous concerne, mon binôme et moi, nous devons développer un réseaux de neurones capables de reconnaître différents états de stimulation d'une personne atteinte de la maladie d'Alzheimer.
 C'est après vous avoir présenté l'ensemble de nos lectures sur l'utilité et l'implémentation des réseaux de neurones et des interfaces cerveaux-machines que nous vous présenterons l'architecture de notre application, ce après quoi nous évoquerons les résultats que nous attendons de ce projets et les perspectives futures de développement.