\part*{Introduction}

 Durant notre première année de master en informatique à l'université Paris 8 Vincennes - Saint-Denis, nous devions développer des applications au cours d'un module intitulé "Projets tuteurés".
 Ce travail serait effectué sur toute l'année et doit faire l'objet d'une présentation en plus de ce rapport-ci. Nous devions choisir une thématique parmi celles proposées.
 Ensuite, notre tuteur attitré nous attribuait un projet en nous expliquant ce qu'il attendait de nous. En ce qui nous concerne, nous devons développer un réseaux de neurones capables de reconnaître différents états de stimulation d'une personne.
 C'est après vous avoir présenté certaines généralité touchant au BCI, puis sur l'utilité et l'implémentation des réseaux de neurones et des interfaces cerveaux-machines que nous vous présenterons l'architecture de notre application. Ce après quoi nous exposerons l'état de l'art en ce qui concerne la détection et la classification d'activité cérébrale. Puis, nous expliquerons le fonctionnement du systèmes. Ensuite, nous détaillerons l'architecture de l'application. Pour enfin conclure sur les objectifs atteint et le prévision future .  