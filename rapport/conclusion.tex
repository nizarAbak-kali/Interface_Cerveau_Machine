\part{Conclusion}
	
	\section{Objectif atteint} % (fold)

		Toutes les fonctions mathématiques ont été développées ainsi que les fonctions concernant l'espace d'Hilbert et la Fast Fourrier Transform. Par ailleurs, nous avons terminé la classe de récupération du fichier qui sera modifiée une fois en possession de la base de données sur laquelle nous travaillerons. Nous avons également développé une classe permettant de réaliser les graphiques qui représenteront les résultats de notre programme avec cette base de données. 	
	
	\label{sec:objectif_atteint}
	
	% section objectif_atteint (end)
	\section{Objectif restant} % (fold)
	\label{sec:objectif_restant}
	
	% section objectif_restant (end)
	Pour conclure, nous pouvons dire que ce projet est très intéressant car il nous a permis de découvrir des domaines concrets insoupçonnés où l'informatique pouvait intervenir afin de faciliter la vie, notamment des personnes lourdement handicapées. Nous n'avons pas encore fini mais pensons vite arriver à nos fins si nous travaillons régulièrement, à raison de une à deux fois par semaine afin de vous fournir un projet fonctionnel lors de la deuxième soutenance, sachant que l'on espère terminer début mars.
	Il nous reste à développer les caractéristiques temporelles ainsi que la classe du signal EEG. 