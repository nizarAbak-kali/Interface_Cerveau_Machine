\part{Conclusion}
	
	\section{Objectifs atteints} % (fold)

		Toutes les fonctions mathématiques sur les complexes ont été développées ainsi que les fonctions concernant l'espace d'Hilbert et la Fast Fourrier Transform. Par ailleurs, nous avons terminé la classe de récupération du fichier qui sera modifiée une fois en possession de la base de données sur laquelle nous travaillerons. Nous avons également développé une classe permettant d'afficher les signaux pour voir ce qu'ils donnent. Enfin, nous avons développé le réseau de neurones ainsi que sa structure. 	
	
	\label{sec:objectif_atteint}
	
	% section objectif_atteint (end)
	\section{Objectifs restants et prévisions} % (fold)
	\label{sec:objectif_restant}
	
	% section objectif_restant (end)
	Pour conclure, nous pouvons dire que ce projet est très intéressant car il nous a permis de découvrir des domaines concrets insoupçonnés où l'informatique pouvait intervenir afin de faciliter la vie, notamment des personnes lourdement handicapées. Nous n'avons toutefois pas mené à bien nos objectifs car nous nous sommes mal organisés. Du coup, nous avons tout simplement essayé de mettre en oeuvre le réseau de neurones et son fonctionnement sur les données alors que nous aurions du, dans l'idéal, axer notre projet sur la partie purement signal. Nous espérons pouvoir vous fournir un réseau de neurones fonctionnels, quelques petits réglages restant à faire.