\part*{Référence} % (fold)
\label{prt:référence_}
Dans cette partie nous exposons les diffenrentes sources que nous avons trouvées utiles lors de la première
partie du projet tuteuré .
\begin{itemize}
  	\item[-]https://en.wikipedia.org/wiki/Feedforward_neural_network;
  	\item[-]https://fr.wikipedia.org/wiki/Gradient;
  	\item[-]https://fr.wikipedia.org/wiki/Algorithme_du_gradient;
  	\item[-]https://fr.wikipedia.org/wiki/R%C3%A9tropropagation_du_gradient;
  	\item[-]https://takinginitiative.wordpress.com/2008/04/03/basic-neural-network-tutorial-theory/
  	\item[-]https://takinginitiative.wordpress.com/2008/04/23/basic-neural-network-tutorial-c-implementation-and-source-code/
  	\item[-]https://fr.wikipedia.org/wiki/Interface_neuronale_directe
  	\item[-]http://www.inserm.fr/thematiques/technologies-pour-la-sante/dossiers-d-information/l-interface-cerveau-machine-ou-agir-par-la-pensee
  	\item[-]http://www.internetactu.net/2012/11/21/interfaces-cerveau-machines-defis-et-promesses/
  	\item[-]https://github.com/marytts/marytts/blob/master/marytts-signalproc/src/main/java/marytts/util/math/FFTMixedRadix.java
  	\item[-]http://www.tangentex.com/FFT.htm
  	\item[-]http://www.javaworld.com/article/2074798/build-ci-sdlc/chart-a-new-course-with-jfreechart.html
  	\item[-]http://introcs.cs.princeton.edu/java/97data/FFT.java.html
  	\item[-]https://moodle.umons.ac.be/mod/resource/view.php?id=11577
  	\item[-]http://www.java2s.com/Code/Java/Chart/JFreeChartLineChartDemo1.htm
  	\item[-]http://jcharts.sourceforge.net/samples/scatter.html
  	\item[-]https://en.wikipedia.org/wiki/Steady_state_visually_evoked_potential
  	\item[-]https://fr.wikipedia.org/wiki/P300
  \end{itemize}  
% part référence_ (end)