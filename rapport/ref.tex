\part{Références} % (fold)
\label{prt:référence_}
Dans cette partie nous exposons les différentes sources que nous avons trouvées utiles lors de la première partie du projet tuteuré .

\begin{itemize}
	\item [-] "ON THE USE OF TIME-FREQUENCY FEATURES FOR DETECTING AND CLASSIFYING EPILEPTIC SEIZURE ACTITIES IN NON-STATIONNARY EEG SIGNALS" - Larbi Boubchir, Somaya Al-Maadeed et Ahmed Bourdane. 
  	\item [-] \url{https://en.wikipedia.org/wiki/Feedforward_neural_network;}
  	\item [-] \url{https://fr.wikipedia.org/wiki/Gradient;}
  	\item [-] \url{https://fr.wikipedia.org/wiki/Algorithme_du_gradient;}
  	\item [-] \url{https://fr.wikipedia.org/wiki/R\%C3\%A9tropropagation_du_gradient;}
  	\item [-] \url{https://takinginitiative.wordpress.com/2008/04/03/basic-neural-network-tutorial-theory/}
  	\item [-] \url{https://takinginitiative.wordpress.com/2008/04/23/basic-neural-network-tutorial-c-implementation-and-source-code/}
  	\item [-] \url{https://fr.wikipedia.org/wiki/Interface_neuronale_directe}
  	\item [-] \url{http://www.inserm.fr/thematiques/technologies-pour-la-sante/dossiers-d-information/l-interface-cerveau-machine-ou-agir-par-la-pensee}
  	\item [-] \url{http://www.internetactu.net/2012/11/21/interfaces-cerveau-machines-defis-et-promesses/}
  	\item [-] \url{https://github.com/marytts/marytts/blob/master/marytts-signalproc/src/main/java/marytts/util/math/FFTMixedRadix.java}
  	\item [-] \url{http://www.tangentex.com/FFT.htm}
  	\item [-] \url{http://www.javaworld.com/article/2074798/build-ci-sdlc/chart-a-new-course-with-jfreechart.html}
  	\item [-] \url{http://introcs.cs.princeton.edu/java/97data/FFT.java.html}
  	\item [-] \url{https://moodle.umons.ac.be/mod/resource/view.php?id=11577}
  	\item [-] \url{http://www.java2s.com/Code/Java/Chart/JFreeChartLineChartDemo1.htm}
  	\item [-] \url{http://jcharts.sourceforge.net/samples/scatter.html}
  	\item [-] \url{https://en.wikipedia.org/wiki/Steady_state_visually_evoked_potential}
  	\item [-] \url{https://fr.wikipedia.org/wiki/P300}
  	\item [-] \url{http://www.enseignement.polytechnique.fr/informatique/INF431/GrapheX/docgrapheX.pdf}
  	\item [-] \url{http://thierry-leriche-dessirier.developpez.com/tutoriels/java/afficher-graphe-jfreechart-5-min/}
  	\item [-] \url{https://openclassrooms.com/forum/sujet/ouvrir-et-lire-un-fichier-csv-56738}
  	\item [-] \url{http://thierry-leriche-dessirier.developpez.com/tutoriels/java/charger-donnees-fichier-csv-5-min/}
  \end{itemize}
% part référence_ (end)